\documentclass[pra,twocolumn,preprintnumbers,amsmath,amssymb,nofootinbib,floatfix,longbibliography]{revtex4}

\usepackage{graphicx, bm, tikz, braket, mathrsfs, comment}
\usepackage[breaklinks]{hyperref}
\usepackage{gensymb} %\degree

\makeatletter

\def\graphicscale{\twocolumn@sw{0.3}{0.4}}
\def\graphicthreescale{\twocolumn@sw{0.3}{0.4}}
\newcommand{\rev}[1]{\textcolor{red}{#1}}

\begin{document}

\title{APPENDIX}

\author{}
\affiliation{}

%\date{}

\begin{abstract}
\end{abstract}
\maketitle

\appendix

\section{Implementation Algorithm}

In the following, we will provide the details of the
computational method used to simulated the quantum system.
It can be split in three different macro-area: the initial
Gibbs state, the unitary dynamics and the general thermal
dissipation dynamics.

\subsection{Initial Gibbs State}

As the first point of the protocol says, our quantum system
starts from the Gibbs thermal state associated with the
initial temperature $T_i$. The Gibbs density matrix
operator in this case, is expressed as:
\begin{align}
	\label{Gibbs}
	\hat \rho_i  = \frac{1}{\cal Z} \,\,&
	e^{-\beta_i \hat H_K(\mu)} \,\,,\\
    {\cal Z} =
	{\rm Tr} \,\,e^{-\beta_i \hat H_K(\mu)} \,\,, & \qquad
	\beta_i  = \frac{1}{T_i} \,\,.
\end{align}
To extract the correlation functions from this initial
condition, we apply a Bogoliubov transformation on the
Kitaev Hamiltonian $\hat H_K(\mu)$:
\begin{equation}
	\label{HKdiag}
	H_K(\mu)=\sum _{j=1}^L\,\omega _j \,\hat b^\dagger _j\,
	\hat b_j+ \frac{1}{2}\,\Bigr[ -2L\mu - \sum _{r=1}^L
	\omega _r \Bigr] \,\,;
\end{equation}
where the $\omega_k$ is the spectrum of the Bogoliubov
fermion eigenoperators $\hat b_k$. The relations, which
connect the previous basis with the Bogoliubov one, can be
written as:
\begin{equation}
  \label{transBogol}
  \hat c_x = \sum_{k=1}^L A_{xk} b_k + B_{xk} b^\dagger_k
  \,\,,
\end{equation}
where $A$ and $B$ are matrices associated with the basis
change. This is a simply linear transformation which pass
from the Bogoliubov operator space to the site operator one.\\

  We can assume the quasi-particles associated with $b_k$
  as a free fermion gas in which the partition function is:
  \begin{equation}
    \label{Partition}
    {\cal Z} = {\rm Tr} \,\, e^{-\beta_i
      \sum_j \omega _j \,\hat b^\dagger _j\, \hat b_j}\,\,;
  \end{equation}
  less than constant terms.\\
  In this state, the
  correlation functions $\braket{b^\dagger_k b_k}$
  can be derived by the following relations:
  \begin{align}
    \label{Zcorr}
    \braket{b^\dagger_k b_k} =
    {\rm Tr} \Biggr[ \frac{b^\dagger_k b_k}
                    {\cal Z}e^{-\beta_i \hat H_K} \Biggr]
    = -\frac{1}{\beta_i}\partial_{\omega_k}\ln{\cal Z}\,\,;
  \end{align}
  where we use that the Gibbs state is equal to
  $\hat \rho = e^{-\beta_i\hat H_K(\mu)}/{\cal Z}$. To
  compute the ${\cal Z}$ function, we consider all the
  basis vector in the Hilbert space associated with the
  free quasi-particle fermions. In this way:
  \begin{equation}
    {\cal Z} = \prod _{j=1}^L
    \Bigl( 1 + e^{-\beta_i \omega_j} \Bigl) \,\,;
  \end{equation}
  and from Eq. (\ref{Zcorr}):
  \begin{align}
    \braket{b^\dagger_k b_k}= &
    -\frac{1}{\beta_i}\partial_{\omega_k} \sum _{j=1}^L
    \ln\Bigl( 1 + e^{-\beta_i \omega_j} \Bigl)= \\
    = &  \frac{e^{-\beta_i \omega_k}}
            { 1 + e^{-\beta_i \omega_k} } =
    \frac{1}{1 + e^{\beta_i \omega_k}}
    \,\,;
  \end{align}
  where $f(\omega_k) = 1 / (1 + e^{\beta_i \omega_k})$ is
  the statistical Fermi-Dirac distribution function. The
  non-diagonal terms $\braket{b^\dagger_q b_k}$ of the
  Bogoliubov correlation functions are all equal to zero,
  since $H_K$ is diagonal in the basis $\{\hat b_k\}$.\\

Finally, to evaluate the correlation function $C(x,y)$ and
$P(x,y)$, we use the Eq. (\ref{transBogol}) in the
definition of the previous 2-points functions and we
achive:
\begin{align}
  \label{initcorr}
   C(x,y) = \sum_{k,q=1}^L
  A^*_{xk}A_{yq} \braket{b^\dagger_k b_q}
  + B^*_{xk} B_{yq} \braket{b_k b^\dagger_q} + \notag\\
  A^*_{xk} B_{yq} \braket{b^\dagger_k b^\dagger_q} +
  B^*_{xk} A_{yq} \braket{b_k b_q}
  \,\,;
\end{align}
and a similar relation for $P(x,y)$.

\subsection{Unitary Dynamics}

Starting from the Gibbs state introduced above, at $t=0$,
we perform a quench protocol on the Kitaev Hamiltonian,
from an initial value $\mu = \mu_i$ to a final one $\mu =
\mu_f$. This process lead to a reformulation of the
Bogoliubov bases in which the time evolution operator
acts.\\This operator is unitary because the state is not in
contact with any thermal bath. The correlation functions
are achieved from the solution, in the Heisenberg picture,
of:
\begin{align}
   \label{EQUnitary}
   \frac{d O_{H}(t) }{d t}  &= i\,\Bigr[\hat H,
      O_{H}(t) \Bigr] ;\\
    %O_H &= \Bigl(c^\dagger_x c_y\Bigl)_H\,\,.
\end{align}
where the initial condition are given by Eq.
(\ref{initcorr}).\\
We substitute the variable $O_{H}$ with the operators
associated with these three observable:
\begin{align}
C(x,y,t) & =  2\,{\rm Re}\Bigr[
\,{\mathscr{C}_{x,y}}(t) \,\Bigr]        \,\,, \\
P(x,y,t) & = 2\,{\rm Re}\Bigr[
\,{\mathscr{P}_{x,y}}(t) \Bigr]   \,\,,\\
D(t) & = \sum_x \mathscr{C}_{x,x} \,\,.
\end{align}
where we have defined the following auxiliary
variables $\mathscr{C}$ and $\mathscr{P}$:
\begin{align}
  \label{RedCorr}
  \mathscr{C}_{x,y} &= {\rm Tr}\Bigr[\rho \hat c^\dagger_x
    \hat c_y\Bigr]\,\,,\\
  \mathscr{P}_{x,y} &= {\rm Tr}\Bigr[\rho \hat c^\dagger_x
    \hat c^\dagger_y\Bigr]\,\,.
\end{align}
Resolving the Eq. (\ref{EQUnitary}) for this observable,
their time-evolution can be expressed in the following
way~\cite{TV21}:
\begin{align}
\frac{d\mathscr{C}_{x,y}}{dt}& =
i\,\bigr[\mathscr{C}_{x,y+1} - \mathscr{C}_{x-1,y} +
\mathscr{C}_{x,y-1} - \mathscr{C}_{x+1,y} \bigr] - \notag\\
-i\, \Bigl(& \mathscr{P}_{y,x-1}^\dagger -
\mathscr{P}_{y,x+1}^\dagger \Bigl)  + i\, \Bigl(
\mathscr{P}_{x,y-1} - \mathscr{P}_{x,y+1} \Bigl) \,\,, \\
\frac{d\mathscr{P}_{x,y}}{dt} &=
-i\,\bigr[\mathscr{P}_{x,y+1} + \mathscr{P}_{x+1,y}+
\mathscr{P}_{x,y-1} + \mathscr{P}_{x-1,y} \bigr] - \notag\\
&- 2\,i\,\mu_f  \,\mathscr{P}_{x,y}-i\,\Bigl(
\delta _{x-1,\,y} - \delta _{x+1,\,y} \Bigl) - \notag \\
 &- i\, \Bigl( \mathscr{C}_{x,y-1} -
\mathscr{C}_{y,x-1} - \mathscr{C}_{x,y+1}
+ \mathscr{C}_{y,x+1} \Bigl) \,\,.
\end{align}
The initial condition on $\mathscr{C}_{x,y}$ and
$\mathscr{P}_{x,y}$ is obtained by the relations
(\ref{transBogol}) and (\ref{RedCorr}), where the
Bogoliubov correlations are associated with the initial
Gibbs state.

\subsection{Thermal Dissipation Dynamics}

Now, in the case in which we turn on a homogeneous thermal
dissipation mechanism, the time evolution is not more
unitary, but it is the interplay between two processes:
\begin{itemize}
  \item the Quench protocol;
  \item the dissipation arising from equal thermal baths
  with temperature $T_b$.
  In this case, each system site is coupled with uniform
  and identical thermal baths with temperature $T_b$; using
  the Born-Markov and secular approximation, the time
  derivative of the density matrix $\rho$ obeys to the
  Lindblad master equation~\cite{BP07}.
\end{itemize}
In this general case, the time-evolution equation of the
operators $O_H$ in the Heisenberg picture presents an
additional dissipation term. In details, the thermalization
process of $O_H(t)$ is obtained by the resolution of the
following master equation~\cite{PCD22, DR21}:
\begin{align}
   \label{EQLindblad}
   \frac{d O_{H}(t) }{d t}  &= i\,\Bigr[\hat H,
      O_{H}(t) \Bigr] + \mathbb{D}[O_{H}(t)]\,\,,\\
  \mathbb{D}[O_H(t)] =
  \,\,\gamma \sum _k &
  f(\omega_k)\,\biggr[2\hat b_{k}^\dagger O_H(t)\hat b_{k}-
  \Bigl\{ O_H(t), \, \hat b_{k} \hat b_{k}^\dagger \Bigl\}
  \biggr]  + \notag\\
  +\gamma \sum_k  (1-f(\omega_k)) &
  \,\biggr[ 2\hat b_{k} O_H(t) \hat b_{k}^\dagger - \Bigl\{
  O_H(t), \, \hat b_{k}^\dagger \hat b_{k} \Bigl\} \biggr]
  \,\,.
\end{align}
The initial state, as in the previous case, is the Gibbs
state in thermal equilibrium at the initial temperature
$T_i$. This state corresponds with the steady state
solution of the Eq. (\ref{EQLindblad}) with fixed
$H(\mu_i)$. Then, the change of the chemical potential to
$\mu_f$ with a different Hamiltonian leads to a
redefinition of the Bogoliubov operators. We call
$\{b'_k\}$ the operators which diagonalizes $H(\mu_f)$:
\begin{equation}
  \label{quenchHdiag}
  H(\mu_f)=\sum _{k=1}^L\,\omega'_k \,\hat b'^\dagger _k\,
  \hat b'_k+ \frac{1}{2}\,\Bigr[ -2L\mu_f - \sum _{r=1}^L
  \omega'_r \Bigr] \,\,;
\end{equation}
where $\{\omega'_k\}$ is the new Bogoliubov spectrum
associated with $H(\mu_f)$.\\
To evaluate the Bogoliubov correlations in this new base
$\{b'_k\}$, we resolve the Eq. (\ref{EQLindblad}) in the
operators $\{b'_k\}$:
\begin{align}
  \label{EQLindbladprime}
  \braket{b'^\dagger_k b'_k}(t) =& f(\omega'_k) \,
  \Bigr[ 1 - \exp\Bigl( -2 \gamma t \Bigl) \Bigr]+ \notag\\
    & + \braket{b'^\dagger_k b'_k}_0
    \exp\Bigl( -2 \gamma t \Bigl) \,\,;\\
  \braket{b'^\dagger_k b'_q}(t) =&
  \braket{b'^\dagger_k b'_q}_0
  \exp\Bigl( i(\omega'_k - \omega'_q)t-2 \gamma t \Bigl)
  \,\,;\\
  \braket{b'^\dagger_k b'^\dagger_q}(t) =&
  \braket{b'^\dagger_k b'^\dagger_q}_0
  \exp\Bigl( i(\omega'_k + \omega'_q)t-2 \gamma t \Bigl)
  \,\,;\\
  \braket{b'_k b'_q}(t) =&
  \braket{b'_k b'_q}_0
  \exp\Bigl( -i(\omega'_k + \omega'_q)t-2 \gamma t \Bigl)
  \,\,;\\
\end{align}
The initial values $\braket{b'^\dagger_k b'_q}_0$ of the
correlations is computed on the initial Gibbs state and
it is given by changing the base from
$\{b_k \}$ to $\{b'_k\}$.

Finally, we use at each step the relations
(\ref{initcorr}) to pass from the Bogoliubov operator to
the correlation functions $C(x,y)$ and $P(x,y)$.

\bibliography{refs.bib}


\end{document}
